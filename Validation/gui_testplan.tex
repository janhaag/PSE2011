\section{GUI Testplan}
\subsection{Menubar}
\subsubsection{"`File"' $\rightarrow$ "`New"'}
\begin{enumerate}
\item "Offnen einer neuen Datei
\begin{itemize}
\item Erwartetes Ereignis: Der Inhalt des Editors wird ohne zu speichern gel"oscht, Breakpoints werden entfernt und die Konsolen geleert. 
\item Status: \textcolor{blue}{BEHOBEN} \\
Breakpoints werden nicht entfernt, Konsole nicht geleert
\end{itemize}
\end{enumerate}
\subsubsection{"`File"' $\rightarrow$ "`Load"'}
\begin{enumerate}
\item Laden einer nichtexistenten Datei
\begin{itemize}
\item Erwartetes Ereignis: Die Datei wird nicht geladen. 
\item Status: \textcolor{green}{OK} 
\end{itemize}
\item Laden einer Datei, die vom Programm erzeugt wurde oder einer txt-Datei
\begin{itemize}
\item Erwartetes Ereignis: Die ausgew"ahlte Datei wird in den Editor geladen. 
\item Status: \textcolor{green}{OK}
\end{itemize}
\item Laden einer Datei, die nicht vom Programm erzeugt wurde und keine txt-Datei ist
\begin{itemize}
\item Erwartetes Ereignis: Die Datei wird nicht im Filedialog angezeigt. 
\item Status: \textcolor{blue}{BEHOBEN} \\
Die Datei wird angezeigt und Programm h"angt sich auf, wenn versucht wird, diese zu laden
\end{itemize}
\end{enumerate}
\subsubsection{"`File"' $\rightarrow$ "`Save"'}
\begin{enumerate}
\item Speichern in einer nichtexistenten Datei
\begin{itemize}
\item Erwartetes Ereignis: Eine wp-Datei wird erzeugt, der Inhalt des Editors darin gespreichert. 
\item Status: \textcolor{green}{OK} 
\end{itemize}
\item Speichern in einer beliebigen Datei
\begin{itemize}
\item Erwartetes Ereignis: Der Inhalt der Datei wird durch den des Editors ersetzt. 
\item Status: \textcolor{green}{OK} 
\end{itemize}
\end{enumerate}
\subsubsection{"`File"' $\rightarrow$ "`Exit"'}
\begin{enumerate}
\item Beenden des Programms
\begin{itemize}
\item Erwartetes Ereignis: Das Programm wird sofort beendet. 
\item Status: \textcolor{green}{OK}
\end{itemize}
\end{enumerate}
\subsubsection{"`Edit"' $\rightarrow$ "`Undo"'}
\begin{enumerate}
\item R"uckg"angigmachen des zuletzt eingetippten Zeichen
\begin{itemize}
\item Erwartetes Ereignis: Das zuletzt eingetippte Zeichen wird gel"oscht. 
\item Status: \textcolor{green}{OK}
\end{itemize}
\item Beliebige Wiederholung von Punkt 1
\begin{itemize}
\item Erwartetes Ereignis: Die zuletzt eingetippten Zeichen werden gel"oscht. 
\item Status: \textcolor{green}{OK}
\end{itemize}
\item R"uckg"angigmachen des zuletzt gel"oschten Zeichen
\begin{itemize}
\item Erwartetes Ereignis: Das zuletzt gel"oschte Zeichen wird wieder hergestellt. 
\item Status: \textcolor{blue}{BEHOBEN} \\
Das zuletzt gel"oschte Zeichen wird nicht wieder hergestellt
\end{itemize}
\item Beliebige Wiederholung von Punkt 3
\begin{itemize}
\item Erwartetes Ereignis: Die zuletzt gel"oschten Zeichen werden wieder hergestellt. 
\item Status: \textcolor{green}{OK}
\end{itemize}
\item R"uckg"angigmachen der letzten Paste-Aktion
\begin{itemize}
\item Erwartetes Ereignis: Die zuletzt eingef"ugte Zeichenkette wird gel"oscht. 
\item Status: \textcolor{green}{OK}
\end{itemize}
\item Beliebige Wiederholung von Punkt 5
\begin{itemize}
\item Erwartetes Ereignis: Die zuletzt eingef"ugten Zeichenketten werden gel"oscht. 
\item Status: \textcolor{green}{OK}
\end{itemize}
\item R"uckg"angigmachen der letzten Cut-Aktion
\begin{itemize}
\item Erwartetes Ereignis: Die zuletzt gel"oschte Zeichenkette wird wieder hergestellt. 
\item Status: \textcolor{green}{OK}
\end{itemize}
\item Beliebige Wiederholung von Punkt 7
\begin{itemize}
\item Erwartetes Ereignis: Die zuletzt gel"oschten Zeichenketten werden wieder hergestellt. 
\item Status: \textcolor{green}{OK}
\end{itemize}
\item R"uckg"angigmachen der Funktion "`File"' $\rightarrow$ "`New"'
\begin{itemize}
\item Erwartetes Ereignis: Der alte Inhalt des Editors wird wieder hergestellt. 
\item Status: \textcolor{green}{OK}
\end{itemize}
\item R"uckg"angigmachen der Funktion "`File"' $\rightarrow$ "`Load"'
\begin{itemize}
\item Erwartetes Ereignis: Der alte Inhalt des Editors wird wieder hergestellt. 
\item Status: \textcolor{green}{OK}
\end{itemize}
\item Undo, obwohl noch keine Aktion ausgef"uhrt wurde
\begin{itemize}
\item Erwartetes Ereignis: Es passiert nichts. 
\item Status: \textcolor{green}{OK}
\end{itemize}
\end{enumerate}
\subsubsection{"`Edit"' $\rightarrow$ "`Redo"'}
\begin{enumerate}
\item R"uckg"angigmachen der letzten Undo-Aktion
\begin{itemize}
\item Erwartetes Ereignis: Die r"uckg"angig gemachte Aktion wird hergestellt. 
\item Status: \textcolor{green}{OK}
\end{itemize}
\item Beliebige Wiederholung von Punkt 1
\begin{itemize}
\item Erwartetes Ereignis: Die r"uckg"angig gemachten Aktionen werden hergestellt. 
\item Status: \textcolor{green}{OK}
\end{itemize}
\item Redo, obwohl noch kein Undo ausgef"uhrt wurde
\begin{itemize}
\item Erwartetes Ereignis: Es passiert nichts. 
\item Status: \textcolor{green}{OK}
\end{itemize}
\end{enumerate}
\subsubsection{"`Edit"' $\rightarrow$ "`Cut"'}
\begin{enumerate}
\item L"oschen der markierten Zeichenkette
\begin{itemize}
\item Erwartetes Ereignis: Die markierte Zeichenkette wird gel"oscht. 
\item Status: \textcolor{green}{OK}
\end{itemize}
\item Cut ohne markierte Zeichenkette
\begin{itemize}
\item Erwartetes Ereignis: Es passiert nichts. 
\item Status: \textcolor{blue}{BEHOBEN} \\
Programm st"urzt ab.
\end{itemize}
\end{enumerate}
\subsubsection{"`Edit"' $\rightarrow$ "`Copy"'}
\begin{enumerate}
\item Kopieren der markierten Zeichenkette
\begin{itemize}
\item Erwartetes Ereignis: Die markierte Zeichenkette wird zum Kopieren gespeichert. 
\item Status: \textcolor{green}{OK}
\end{itemize}
\item Beliebige Wiederholung von Punkt 1
\begin{itemize}
\item Erwartetes Ereignis: Die zuletzt kopierte Zeichenkette wird gespeichert. 
\item Status: \textcolor{green}{OK}
\end{itemize}
\item Copy ohne markierte Zeichenkette
\begin{itemize}
\item Erwartetes Ereignis: Es passiert nichts. 
\item Status: \textcolor{blue}{BEHOBEN} \\
Programm st"urzt ab.
\end{itemize}
\end{enumerate}
\subsubsection{"`Edit"' $\rightarrow$ "`Paste"'}
\begin{enumerate}
\item Einf"ugen der aus dem Programm kopierten Zeichenkette
\begin{itemize}
\item Erwartetes Ereignis: Die kopierte Zeichenkette wird im Editor eingef"ugt. 
\item Status: \textcolor{green}{OK}
\end{itemize}
\item Einf"ugen der aus einem anderen Programm kopierten Zeichenkette
\begin{itemize}
\item Erwartetes Ereignis: Die kopierte Zeichenkette wird im Editor eingef"ugt. 
\item Status: \textcolor{green}{OK}
\end{itemize}
\item Beliebige Wiederholung von Punkt 1 oder 2
\begin{itemize}
\item Erwartetes Ereignis: Die kopierte Zeichenkette wird jedes Mal im Editor eingef"ugt. 
\item Status: \textcolor{green}{OK}
\end{itemize}
\end{enumerate}
\subsubsection{"`Edit"' $\rightarrow$ "`Settings"'}
\begin{enumerate}
\item "Offnen des Settingsfensters
\begin{itemize}
\item Erwartetes Ereignis: Das Fenster zur Einstellung von Z3-Settings wird ge"offnet. 
\item Status: \textcolor{green}{OK}
\end{itemize}
\end{enumerate}
\subsubsection{"`Run"' $\rightarrow$ "`Random Test"'}
\begin{enumerate}
\item "Offnen des Randomtestfensters
\begin{itemize}
\item Erwartetes Ereignis: Das Fenster f"ur Randomtests wird ge"offnet. 
\item Status: \textcolor{green}{OK}
\end{itemize}
\end{enumerate}
\subsubsection{"`Help"' $\rightarrow$ "`Help"'}
\begin{enumerate}
\item "Offnen des Helpfensters und Anzeigen der Helpdokumentation
\begin{itemize}
\item Erwartetes Ereignis: Die Helpdokumentation wird ge"offnet. 
\item Status: \textcolor{green}{OK}
\end{itemize}
\end{enumerate}
\subsubsection{"`Help"' $\rightarrow$ "`About"'}
\begin{enumerate}
\item "Offnen des Aboutfensters
\begin{itemize}
\item Erwartetes Ereignis: Das Aboutfenster wird ge"offnet. 
\item Status: \textcolor{green}{OK}
\end{itemize}
\end{enumerate}

\subsection{Frames}
\subsubsection{Settingsframe}
\begin{enumerate}
\item Speichern von korrekten Eingaben
\begin{itemize}
\item Erwartetes Ereignis: Es wird eine Erfolgsmeldung ausgegeben und die neuen Eingaben stehen in den entsprechenden Textfeldern.
\item Status: \textcolor{green}{OK}
\end{itemize}
\item Speichern von inkorrekten Eingaben
\begin{itemize}
\item Erwartetes Ereignis: Es wird eine Fehlermeldung ausgegeben und die alten Werte werden wiederhergestellt.  
\item Status: \textcolor{blue}{BEHOBEN} \\
Wenn der Pfad nicht korrekt eingegeben wurde, wird trotzdem die Erfolgsmeldung angezeigt.
\end{itemize}
\item Klick auf "`Close"' Button
\begin{itemize}
\item Erwartetes Ereignis: Das Settingsfenster wird geschlossen.
\item Status: \textcolor{green}{OK}
\end{itemize}
\item Ausf"uhren von 1, 3 und anschlie"sendes "Offnen des Fensters.
\begin{itemize}
\item Erwartetes Ereignis: Die bei 1 eingegebenen neuen Werte stehen immernoch in den entsprechenden Textfeldern.
\item Status: \textcolor{green}{OK}
\end{itemize}
\item Ausf"uhren von 2, 3 und anschlie"sendes "Offnen des Fensters.
\begin{itemize}
\item Erwartetes Ereignis: Die Werte vor der "Anderung stehen immernoch in den entsprechenden Textfeldern.
\item Status: \textcolor{green}{OK}
\end{itemize}
\end{enumerate}
\subsubsection{Helpframe}
\begin{enumerate}
\item Ausw"ahlen der einzelnen Abschnitte
\begin{itemize}
\item Erwartetes Ereignis: Der ausgew"ahlte Abschnitt wird angezeigt.
\item Status: \textcolor{green}{OK}
\end{itemize}
\item Klick auf "`Close"' Button
\begin{itemize}
\item Erwartetes Ereignis: Das Helpfenster wird geschlossen.
\item Status: \textcolor{green}{OK}
\end{itemize}
\end{enumerate}

\subsection{Views}
\subsubsection{Editor}
\begin{enumerate}
\item Eingabe, Modifikation von Quelltext im idle-Zustand
\begin{itemize}
\item Erwartetes Ereignis: Der Inhalt des Editors kann beliebig ver"andert werden, solange es kein Programm l"auft oder pausiert ist. 
\item Status: \textcolor{green}{OK}
\end{itemize}
\item Eingabe, Modifikation von Quelltext im nicht-idle-Zustand
\begin{itemize}
\item Erwartetes Ereignis: Der Inhalt des Editors kann nicht ver"andert werden, solange ein Programm l"auft oder pausiert ist. 
\item Status: \textcolor{green}{OK}
\end{itemize}
\item Eingabe von Keywords und Zahlen
\begin{itemize}
\item Erwartetes Ereignis: Die Keywords "`int, bool, array, true, false, main, while, if, else, return, assert, assume, ensure, invariant"' und Zahlen werden farbig hervorgehoben. 
\item Status: \textcolor{green}{OK}
\end{itemize}
\item Setzen oder Entfernen von Statementbreakpoints im nicht-idle-Zustand
\begin{itemize}
\item Erwartetes Ereignis: Breakpoints k"onnen nicht gesetzt oder entfernt werden, solange ein Programm l"auft oder pausiert ist. 
\item Status: \textcolor{green}{OK}
\end{itemize}
\item Setzen von Statementbreakpoints im idle-Zustand
\begin{itemize}
\item Erwartetes Ereignis: Statementbreakpoints k"onnen nur gesetzt werden, wenn in der Zeile ein Statement steht.
\item Status: \textcolor{green}{OK}
\end{itemize}
\item Entfernen von Statementbreakpoints im idle-Zustand
\begin{itemize}
\item Erwartetes Ereignis: Breakpoint wird entfernt.
\item Status: \textcolor{blue}{BEHOBEN} \\
Breakpoint kann nicht mehr entfernt werden, wenn die Zeile so modifiziert wurde, dass sie keinen Statement mehr enth"ahlt
\end{itemize}
\end{enumerate}
\subsubsection{Globalbreakpointview}
\begin{enumerate}
\item Einf"ugen, Entfernen, Aktivieren, Deaktivieren von Globalbreakpoints im nicht-idle-Zustand
\begin{itemize}
\item Erwartetes Ereignis: Globalbreakpoints k"onnen nicht ver"andert werden, solange ein Programm l"auft oder pausiert ist. 
\item Status: \textcolor{green}{OK}
\end{itemize}
\item Einf"ugen und Entfernen von syntaktisch und semantisch korrekten Zeichenketten, z.B. Identifier, Integer-, Booleanliteral, Arrayzugriff, Funktionsaufruf, arithmetischer oder boolscher Ausdruck
\begin{itemize}
\item Erwartetes Ereignis: Der Breakpoint wird eingef"ugt bzw. entfernt. 
\item Status: \textcolor{green}{OK}
\end{itemize}
\item Einf"ugen von syntaktisch oder semantisch inkorrekten Zeichenketten, z.B. Deklaration, Zuweisung, Spezifikation, If-, While-, Return-Anweisung, Ausdruck mit Quantoren
\begin{itemize}
\item Erwartetes Ereignis: Der Breakpoint wird nicht eingef"ugt. 
\item Status: \textcolor{blue}{BEHOBEN} \\
Einf"ugen nach einem korrekt eingef"ugten Breakpoint bringt das Programm zum Absturz.
\end{itemize}
\end{enumerate}
\subsubsection{Helpbox}
\begin{enumerate}
\item Es wird eine Stringkette eingegeben und nach Hilfe gesucht
\begin{itemize}
\item Erwartetes Ereignis: Der zur Stringkette am relevanteste Abschnitt wird in der Helpbox angezeigt. 
\item Status: \textcolor{blue}{BEHOBEN} \\
Wenn nach "`else"' gesucht wird, erscheint die Einleitung
\end{itemize}
\end{enumerate}

\subsection{Testprogramme}
\subsubsection{Leeres Programm}
Programm mit leerem String
\begin{enumerate}
\item Check Syntax
\begin{itemize}
\item Erwartetes Ereignis: Fehlermeldung in der Errorkonsole, dass das Programm keine main-Methode besitzt. 
\item Status: \textcolor{green}{OK}
\end{itemize}
\item Run/Single Step
\begin{itemize}
\item Erwartetes Ereignis: Die gleiche Fehlermeldung in der Errorkonsole wie in Punkt 1. 
\item Status: \textcolor{green}{OK}
\end{itemize}
\item Validate
\begin{itemize}
\item Erwartetes Ereignis: Es passiert nichts. 
\item Status: \textcolor{green}{OK}
\end{itemize}
\item Randomtest
\begin{itemize}
\item Erwartetes Ereignis: Es erscheint die Meldung, dass das Programm keine korrekte Syntax besitzt. 
\item Status: \textcolor{green}{OK}
\end{itemize}
\end{enumerate}
\subsubsection{Programm ohne richtige main-Methode}
Programm, in dem es keine main-Methode gibt, die main-Methode sich in einem Statementblock befindet oder die main-Methode return-Statement oder R"uckgabewert hat
\begin{enumerate}
\item Check Syntax
\begin{itemize}
\item Erwartetes Ereignis: Fehlermeldung(en) in der Errorkonsole, dass das Programm keine main-Methode oder Syntaxfehler besitzt.
\item Status: \textcolor{green}{OK}
\end{itemize}
\item Run/Single Step
\begin{itemize}
\item Erwartetes Ereignis: Die gleiche Fehlermeldung in der Errorkonsole wie in Punkt 1. 
\item Status: \textcolor{green}{OK}
\end{itemize}
\item Validate
\begin{itemize}
\item Erwartetes Ereignis: Es passiert nichts. 
\item Status: \textcolor{green}{OK}
\end{itemize}
\item Randomtest
\begin{itemize}
\item Erwartetes Ereignis: Es erscheint die Meldung, dass das Programm keine korrekte Syntax besitzt. 
\item Status: \textcolor{green}{OK}
\end{itemize}
\end{enumerate}
\subsubsection{Programm zum Testen von Breakpoints}
\begin{enumerate}
\item Setzen von Statementbreakpoints an beliebiger Stelle
\begin{itemize}
\item Erwartetes Ereignis: Die Programmausf"uhrung wird angehalten, wenn ein Statementbreakpoint getroffen wurde.
\item Status: \textcolor{green}{OK}
\end{itemize}
\item Setzen von Globalbreakpoints (aktiviert oder deaktiviert)
\begin{itemize}
\item Erwartetes Ereignis: Die Programmausf"uhrung wird angehalten, wenn ein aktiver Globalbreakpoint getroffen wurde.
\item Status: \textcolor{green}{OK}
\end{itemize}
\end{enumerate}
\subsubsection{Programm zum Testen von Randomtests}
\begin{enumerate}
\item Ausf"uhrung mit korrekten Eingaben
\begin{itemize}
\item Erwartetes Ereignis: Es werden Werte aus den angegeben Intervallen ausgew"ahlt und in der Misckonsole angezeigt.
\item Status: \textcolor{green}{OK}
\end{itemize}
\item Ausf"uhrung mit falschen/leeren Eingaben
\begin{itemize}
\item Erwartetes Ereignis: Die Parameter werden alle auf 0 bzw. false gesetzt.
\item Status: \textcolor{green}{OK}
\end{itemize}
\end{enumerate}
\subsubsection{Programm zum Testen von Arrays}
\begin{enumerate}
\item Check Syntax
\begin{itemize}
\item Erwartetes Ereignis: Syntaxfehler werden korrekt angezeigt.
\item Status: \textcolor{red}{FEHLSCHLAG} \\
Es wird manchmal die ungenaue Fehlermeldung "`AST creation not possible!"'zur"uckgegeben
\end{itemize}
\item Run/Single Step
\begin{itemize}
\item Erwartetes Ereignis: Korrekte Ausf"uhrung des Programms und Erkennung von "Uberschreitung der Arraygrenze.
\item Status: \textcolor{green}{OK}
\end{itemize}
\end{enumerate}
\subsubsection{Programm zum Testen von Funktionen}
\begin{enumerate}
\item Check Syntax
\begin{itemize}
\item Erwartetes Ereignis: Syntaxfehler werden korrekt angezeigt.
\item Status: \textcolor{green}{OK}
\end{itemize}
\item Run/Single Step
\begin{itemize}
\item Erwartetes Ereignis: Korrekte Ausf"uhrung des Programms.
\item Status: \textcolor{red}{FEHLSCHLAG} \\
Bei verschachtelten Funktionsaufrufen werden die "au"seren Funktionen "ubersprungen
\end{itemize}
\end{enumerate}
\subsubsection{Programm zum Testen von If-Anweisungen}
\begin{enumerate}
\item Check Syntax
\begin{itemize}
\item Erwartetes Ereignis: Syntaxfehler werden korrekt angezeigt.
\item Status: \textcolor{green}{OK}
\end{itemize}
\item Run/Single Step
\begin{itemize}
\item Erwartetes Ereignis: Korrekte Ausf"uhrung des Programms.
\item Status: \textcolor{green}{OK}
\end{itemize}
\end{enumerate}
\subsubsection{Programm zum Testen von While-Schleifen}
\begin{enumerate}
\item Check Syntax
\begin{itemize}
\item Erwartetes Ereignis: Syntaxfehler werden korrekt angezeigt.
\item Status: \textcolor{green}{OK}
\end{itemize}
\item Run/Single Step
\begin{itemize}
\item Erwartetes Ereignis: Korrekte Ausf"uhrung des Programms.
\item Status: \textcolor{green}{OK}
\end{itemize}
\end{enumerate}
\subsubsection{Programm zum Testen von Operatoren}
\begin{enumerate}
\item Check Syntax
\begin{itemize}
\item Erwartetes Ereignis: Syntaxfehler werden korrekt angezeigt.
\item Status: \textcolor{green}{OK}
\end{itemize}
\item Run/Single Step
\begin{itemize}
\item Erwartetes Ereignis: Korrekte Ausf"uhrung des Programms.
\item Status: \textcolor{green}{OK}
\end{itemize}
\end{enumerate}
\subsubsection{Programm zum Testen von Assertions}
\begin{enumerate}
\item Check Syntax
\begin{itemize}
\item Erwartetes Ereignis: Syntaxfehler werden korrekt angezeigt.
\item Status: \textcolor{green}{OK}
\end{itemize}
\item Run/Single Step
\begin{itemize}
\item Erwartetes Ereignis: Korrekte Ausf"uhrung des Programms und Erkennung von Assertionfailures.
\item Status: \textcolor{green}{OK}
\end{itemize}
\end{enumerate}
\subsubsection{Programm zum Testen von Assumptions}
\begin{enumerate}
\item Check Syntax
\begin{itemize}
\item Erwartetes Ereignis: Syntaxfehler werden korrekt angezeigt.
\item Status: \textcolor{green}{OK}
\end{itemize}
\item Run/Single Step
\begin{itemize}
\item Erwartetes Ereignis: Korrekte Ausf"uhrung des Programms und Erkennung von Assumptionfailures.
\item Status: \textcolor{green}{OK}
\end{itemize}
\end{enumerate}
\subsubsection{Programm zum Testen von Ensures}
\begin{enumerate}
\item Check Syntax
\begin{itemize}
\item Erwartetes Ereignis: Syntaxfehler werden korrekt angezeigt.
\item Status: \textcolor{green}{OK}
\end{itemize}
\item Run/Single Step
\begin{itemize}
\item Erwartetes Ereignis: Korrekte Ausf"uhrung des Programms und Erkennung von Ensurefailures.
\item Status: \textcolor{green}{OK}
\end{itemize}
\end{enumerate}
\subsubsection{Programm zum Testen von Invariants}
\begin{enumerate}
\item Check Syntax
\begin{itemize}
\item Erwartetes Ereignis: Syntaxfehler werden korrekt angezeigt.
\item Status: \textcolor{green}{OK}
\end{itemize}
\item Run/Single Step
\begin{itemize}
\item Erwartetes Ereignis: Korrekte Ausf"uhrung des Programms und Erkennung von Invariantfailures.
\item Status: \textcolor{green}{OK}
\end{itemize}
\end{enumerate}
\subsubsection{Programm zum Testen von Axiomen}
\begin{enumerate}
\item Check Syntax
\begin{itemize}
\item Erwartetes Ereignis: Syntaxfehler werden korrekt angezeigt.
\item Status: \textcolor{red}{FEHLSCHLAG} \\
Es wird manchmal die ungenaue Fehlermeldung "`AST creation not possible!"'zur"uckgegeben
\end{itemize}
\item Run/Single Step
\begin{itemize}
\item Erwartetes Ereignis: Korrekte Ausf"uhrung des Programms, indem die Axiome ignoriert werden.
\item Status: \textcolor{green}{OK}
\end{itemize}
\end{enumerate}